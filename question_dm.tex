\documentclass[a4paper,11pt]{article}

\usepackage[utf8]{inputenc}
\usepackage{times}           
\usepackage[french]{babel}

\usepackage{graphicx}
\usepackage{amsmath, amsthm, amssymb}  
\usepackage{url}

\sloppy

\title{DM d'algo parallèle}
\author{Baptiste Rozière et Alice Pellet-Mary}
\date{\today} 

\begin{document}

\maketitle

\section*{Question 3}


\section*{Question 5}
L'automate est bien définit si on n'a pas dans DEP à la fois nord et sud, ou bien à la fois ouest et est. Si on a nord et sud ou est et ouest, on aura une boucle, chaque case devra se mettre à jour après ses voisins qui devront eux même se mettre à jour après elle.
En revanche, si on n'a pas à la fois nord et sud ou à la fois est et ouest, l'automate sera bien défini car les lignes et colonnes des bords n'ont pas de voisins, donc elles pourront se mettre à jour sans attendre que d'autres cases se mettent à jour.\\
Si la première et dernière ligne sont voisines, on ne peut plus faire de dépendances nord ou sud. Les seuls cas où DEP sera bien défini seront $DEP = \{E\}$, $DEP = \{W\}$ ou $DEP = \emptyset$. Si on a $N$ ou $S$ dans $DEP$, par exemple $N$, la première ligne aura besoin de la dernière pour se mettre à jour, qui aura besoin de l'avant dernière, ..., qui aura besoin de la première. Donc l'automate n'est plus bien défini.\\
Dans le cas où la première et la dernière colonne sont voisines aussi, il n'y a plus aucune dépendance possible. Le seul cas où l'automate sera bien défini est le cas $DEP = \emptyset$.


\section*{Question 6}
On peut trouver un algorithme efficace équivalent à l'algorithme séquentiel si les dépendances sont $DEP = \{N\}$, $DEP = \{W\}$ ou $DEP = \{N, W\}$. Dans les cas où on a $S$ ou $E$ dans les dépendances, on ne peut pas appliquer l'algorithme séquentiel, car comme on va de haut en bas et de gauche à droite, lorsqu'on veut mettre à jour une case, il n'y a que les cases au nord et à l'est qui ont été modifiées. On ne peut donc pas prendre en compte les modifications des cases à l'est et au sud.\\
Dans le cas où $DEP = \{N\}$, on a un algorithme parallèle efficace en pipelinant les calculs... (to be continued)

\section*{Question 8}

\end{document}